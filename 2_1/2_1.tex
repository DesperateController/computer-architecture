\documentclass[12pt, a4paper]{article}

\usepackage[utf8]{inputenc}
\usepackage[russian]{babel}
\parindent 0pt
\parskip 8pt
\usepackage{amsmath}
\usepackage{amssymb}
\usepackage{array}
\usepackage[left=2.3cm, right=2.3cm, top=2.7cm, bottom=2.7cm, bindingoffset=0cm]{geometry} % headheight=0pt,
\usepackage{hyperref}
\usepackage{graphicx}
\usepackage{multicol}
\usepackage{fancyhdr} 
\usepackage{extramarks}
\usepackage[usenames,dvipsnames]{color}
\usepackage{titlesec}
\usepackage{tikz}
\definecolor{grey}{RGB}{128,128,128}

\pagestyle{fancy}
\fancyhf{}
\lhead{Билет № 2.1}
\chead{Архитектура фон Неймана и ее альтернативы}
\rhead{\thepage}
\lfoot{made with Ы}
\cfoot{}
\rfoot{\today}
\renewcommand\headrulewidth{0.4pt}
\renewcommand\footrulewidth{0.4pt}

\titlespacing*{\section}{0pt}{5pt}{0pt}
\titlespacing*{\subsection}{0pt}{5pt}{0pt}
\titlespacing*{\subsubsection}{0pt}{5pt}{0pt}

\begin{document}
\section{Принципы Фон Неймана}
\begin{enumerate}
    \item Двоичная система счисления\\
    Альтернатива: система счисления по основанию 3 (на самом деле даже лучше, т.к. ближе к "идеальной" СС по основанию $e$). Не прижилась т.к. слишком много всего уже работает на двоичной.
    \item Адресность памяти\\
    Доступ к любой ячейке памяти осуществляется за одинаковое время.\\
    Альтернатива: стековая архитектура, машина Тьюринга.
    \item Однородность памяти\\
    Одна память для команд и для данных.\\
    Альтернатива: отдельная память для команд, отдельная для данных (Гарвардская архитектура).
    \item Программное управление\\
    Альтернатива: аппаратное управление
    \item Последовательное выполнение\\
    Команды должны выполняться последовательно.\\
    Альтернатива: многопроцессорные архитектуры и (внезапно) конвеер.
\end{enumerate}
\section{Сравнение гарвардской и фон Неймана}
\subsection{Гарвардская}
\subsubsection{Преимущества}
\begin{itemize}
    \item Быстрее. Раздельные шины под данные и инструкции позволяют читать инструкции одновременно с запросами к памяти.
    \item Код программ защищен от вмешательства других программ.
    \item Память под данные и под инструкции может иметь разные характеристики (разный рамер шины, частоту и т.д.)
\end{itemize}
\subsubsection{Недостатки}
\begin{itemize}
    \item Нужен более сложный и дорогой контроллер памяти
    \item В два раза больше шин тоже дорого
    \item Не можем использовать "лишнюю" память данных под инструкции и наоборот
    \item С точки зрения программиста: не можем модифицировать код. Память команд read-only.
\end{itemize}
\subsection{фон Нейман}
\subsubsection{Преимущества}
\begin{itemize}
    \item Программист может использовать всю доступную память, разделяя инструкции и данные как удобно.
    \item Одна шина вместо двух - проще и дешевле
    \item Контроллер памяти тоже проще
    \item Обращаемся к инструкциям и к данным одинакого
\end{itemize}
\subsubsection{Недостатки}
\begin{itemize}
    \item Одна шина это всё-таки грустно. Структурные хазарды у конвеера и все такое :(
    \item Одна программа может перезаписать другую
\end{itemize}
\section{Модифицированная гарвардская архитектура}
Существует несколько модификаций гарвардской архитектуры.
\subsection{Раздельный кэш}
Используем раздельный кэш первого уровня L1i под инструкции, L1d под данные.
\subsection{Доступ к памяти команд как к данным*}
В этой модификации существуют команды, которые позволяют считывать константы из инструкций в регистры. Пример такой архитектуры - AVR8. Но все равно это головная боль для программиста, почти как труъ-гарвардская
\subsection{Чтение инструкций из памяти данных*}
В общем, такое можно провернуть, но доступ к константам всё равно грустный. Тоже почти труъ-гарвард.
\section{Почему двоичная система счисления?}
Потому что так исторически сложилось. Изначально пытались сделать десятичные компьютеры (аналитическая машина Чарльза Бэббиджа, например). Но это точно неудобно. Оптимальной системой счисления является $e$. Тройка по идее ближе к $e$, но уже слишком много софта написано для двоичной, а железка без софта никому не нужна.
Компьютеры на троичной системе счисления, кстати, вполне существовали. Пример: советский компьютер Сетунь. 
\section{Источники информации}
\begin{enumerate}
    \item Википедия
    \item Какой-то сайтик с \href{http://ithare.com/modified-harvard-architecture-clarifying-confusion/}{\textsc{зайкой}}
\end{enumerate}
\end{document}